
  % 1. a identificação do projeto;
  % 2. a introdução;
  % 3. as atividades desenvolvidas;
  % 4. os resultados obtidos; e
  % 5. as considerações finais.

%% abtex2-modelo-relatorio-tecnico.tex, v-1.9.6 laurocesar
%% Copyright 2012-2016 by abnTeX2 group at http://www.abntex.net.br/ 
%%
%% This work may be distributed and/or modified under the
%% conditions of the LaTeX Project Public License, either version 1.3
%% of this license or (at your option) any later version.
%% The latest version of this license is in
%%   http://www.latex-project.org/lppl.txt
%% and version 1.3 or later is part of all distributions of LaTeX
%% version 2005/12/01 or later.
%%
%% This work has the LPPL maintenance status `maintained'.
%% 
%% The Current Maintainer of this work is the abnTeX2 team, led
%% by Lauro César Araujo. Further information are available on 
%% http://www.abntex.net.br/
%%
%% This work consists of the files abntex2-modelo-relatorio-tecnico.tex,
%% abntex2-modelo-include-comandos and abntex2-modelo-references.bib
%%

% ------------------------------------------------------------------------
% ------------------------------------------------------------------------
% abnTeX2: Modelo de Relatório Técnico/Acadêmico em conformidade com 
% ABNT NBR 10719:2015 Informação e documentação - Relatório técnico e/ou
% científico - Apresentação
% ------------------------------------------------------------------------ 
% ------------------------------------------------------------------------

\documentclass[
  % -- opções da classe memoir --
  12pt,       % tamanho da fonte
  % openright,      % capítulos começam em pág ímpar (insere página vazia caso preciso)
  % twoside,      % para impressão em recto e verso. Oposto a oneside
  oneside,
  a4paper,      % tamanho do papel. 
  % -- opções da classe abntex2 --
  %chapter=TITLE,   % títulos de capítulos convertidos em letras maiúsculas
  %section=TITLE,   % títulos de seções convertidos em letras maiúsculas
  %subsection=TITLE,  % títulos de subseções convertidos em letras maiúsculas
  %subsubsection=TITLE,% títulos de subsubseções convertidos em letras maiúsculas
  % -- opções do pacote babel --
  english,      % idioma adicional para hifenização
  french,       % idioma adicional para hifenização
  spanish,      % idioma adicional para hifenização
  brazil,       % o último idioma é o principal do documento
  ]{abntex2}


% ---
% PACOTES
% ---

% ---
% Pacotes fundamentais 
% ---
\usepackage{lmodern}      % Usa a fonte Latin Modern
\usepackage[T1]{fontenc}    % Selecao de codigos de fonte.
\usepackage[utf8]{inputenc}   % Codificacao do documento (conversão automática dos acentos)
\usepackage{indentfirst}    % Indenta o primeiro parágrafo de cada seção.
\usepackage{color}        % Controle das cores
\usepackage{graphicx}     % Inclusão de gráficos
\usepackage{microtype}      % para melhorias de justificação
% ---

% ---
% Pacotes adicionais, usados no anexo do modelo de folha de identificação
% ---
\usepackage{multicol}
\usepackage{multirow}
% ---
  
% ---
% Pacotes adicionais, usados apenas no âmbito do Modelo Canônico do abnteX2
% ---
\usepackage{lipsum}       % para geração de dummy text
% ---

% ---
% Pacotes de citações
% ---
\usepackage[brazilian,hyperpageref]{backref}   % Paginas com as citações na bibl
\usepackage[alf]{abntex2cite} % Citações padrão ABNT

% --- 
% CONFIGURAÇÕES DE PACOTES
% --- 

% ---
% Configurações do pacote backref
% Usado sem a opção hyperpageref de backref
\renewcommand{\backrefpagesname}{Citado na(s) página(s):~}
% Texto padrão antes do número das páginas
\renewcommand{\backref}{}
% Define os textos da citação
\renewcommand*{\backrefalt}[4]{
  \ifcase #1 %
    Nenhuma citação no texto.%
  \or
    Citado na página #2.%
  \else
    Citado #1 vezes nas páginas #2.%
  \fi}%
% ---

% ---
% Informações de dados para CAPA e FOLHA DE ROSTO
% ---
\titulo{Filtro de Partículas Aplicado à\\Localização de Robôs Móveis\\no Domínio da RoboCup Humanoide}
\autor{Aislan Cesar de Almeida}
\local{São Bernardo do Campo - SP}
\data{Outubro 2017}
\instituicao{%
  Centro Universitário da
  \par
  Fundação Educacional Inaciana Pe. Sabóia de Medeiros - FEI
  \par
  Programa de Pós-Graduação em Engenharia Elétrica
  \par
  Departamento de Inteligência Artificial Aplicada à Automação.}
\tipotrabalho{Relatório de Atividades}
% O preambulo deve conter o tipo do trabalho, o objetivo, 
% o nome da instituição e a área de concentração 
\preambulo{Relatório das atividades desenvolvidas durante o período em que o autor foi bolsista do Conselho Nacional de Desenvolvimento Científico e Tecnológico com o objetivo de obter o título de Mestre de Ciência.}
% ---

% ---
% Configurações de aparência do PDF final

% alterando o aspecto da cor azul
\definecolor{blue}{RGB}{41,5,195}

% informações do PDF
\makeatletter
\hypersetup{
      %pagebackref=true,
    pdftitle={\@title}, 
    pdfauthor={\@author},
      pdfsubject={\imprimirpreambulo},
      pdfcreator={LaTeX with abnTeX2},
    pdfkeywords={abnt}{latex}{abntex}{abntex2}{relatório técnico}, 
    colorlinks=true,          % false: boxed links; true: colored links
      linkcolor=blue,           % color of internal links
      citecolor=blue,           % color of links to bibliography
      filecolor=magenta,          % color of file links
    urlcolor=blue,
    bookmarksdepth=4
}
\makeatother
% --- 

% --- 
% Espaçamentos entre linhas e parágrafos 
% --- 

% O tamanho do parágrafo é dado por:
\setlength{\parindent}{1.3cm}

% Controle do espaçamento entre um parágrafo e outro:
\setlength{\parskip}{0.2cm}  % tente também \onelineskip

% ---
% compila o indice
% ---
\makeindex
% ---

% ----
% Início do documento
% ----
\begin{document}

% Seleciona o idioma do documento (conforme pacotes do babel)
%\selectlanguage{english}
\selectlanguage{brazil}

% Retira espaço extra obsoleto entre as frases.
\frenchspacing 

% ----------------------------------------------------------
% ELEMENTOS PRÉ-TEXTUAIS
% ----------------------------------------------------------
% \pretextual

% ---
% Capa
% ---
\imprimircapa
% ---

% ---
% Folha de rosto
% (o * indica que haverá a ficha bibliográfica)
% ---
\imprimirfolhaderosto*
% ---

% ---
% Anverso da folha de rosto:
% ---

% {
% \ABNTEXchapterfont

% \vspace*{\fill}

% Conforme a ABNT NBR 10719:2015, seção 4.2.1.1.1, o anverso da folha de rosto
% deve conter:

% \begin{alineas}
%   \item nome do órgão ou entidade responsável que solicitou ou gerou o
%    relatório; 
%   \item título do projeto, programa ou plano que o relatório está relacionado;
%   \item título do relatório;
%   \item subtítulo, se houver, deve ser precedido de dois pontos, evidenciando a
%    sua subordinação ao título. O relatório em vários volumes deve ter um título
%    geral. Além deste, cada volume pode ter um título específico; 
%   \item número do volume, se houver mais de um, deve constar em cada folha de
%    rosto a especificação do respectivo volume, em algarismo arábico; 
%   \item código de identificação, se houver, recomenda-se que seja formado
%    pela sigla da instituição, indicação da categoria do relatório, data,
%    indicação do assunto e número sequencial do relatório na série; 
%   \item classificação de segurança. Todos os órgãos, privados ou públicos, que
%    desenvolvam pesquisa de interesse nacional de conteúdo sigiloso, devem
%     informar a classificação adequada, conforme a legislação em vigor; 
%   \item nome do autor ou autor-entidade. O título e a qualificação ou a função
%    do autor podem ser incluídos, pois servem para indicar sua autoridade no
%    assunto. Caso a instituição que solicitou o relatório seja a mesma que o
%    gerou, suprime-se o nome da instituição no campo de autoria; 
%   \item local (cidade) da instituição responsável e/ou solicitante; NOTA: No
%    caso de cidades homônimas, recomenda-se o acréscimo da sigla da unidade da
%    federação.
%   \item ano de publicação, de acordo com o calendário universal (gregoriano),
%   deve ser apresentado em algarismos arábicos.
% \end{alineas}

% \vspace*{\fill}
% }

% ---
% Agradecimentos
% ---
% \begin{agradecimentos}
% O agradecimento principal é direcionado a Youssef Cherem, autor do
% \nameref{formulado-identificacao} (\autopageref{formulado-identificacao}).

% Os agradecimentos especiais são direcionados ao Centro de Pesquisa em
% Arquitetura da Informação\footnote{\url{http://www.cpai.unb.br/}} da Universidade de
% Brasília (CPAI), ao grupo de usuários
% \emph{latex-br}\footnote{\url{http://groups.google.com/group/latex-br}} e aos
% novos voluntários do grupo
% \emph{\abnTeX}\footnote{\url{http://groups.google.com/group/abntex2} e
% \url{http://www.abntex.net.br/}}~que contribuíram e que ainda
% contribuirão para a evolução do abn\TeX.

% \end{agradecimentos}
% ---

% ---
% RESUMO
% ---

% resumo na língua vernácula (obrigatório)
\setlength{\absparsep}{10pt} % ajusta o espaçamento dos parágrafos do resumo
\begin{resumo}
  Este relatório apresenta as atividades desempenhadas pelo autor durante a vigência da bolsa auxílio do programa de mestrado, concedida pelo Conselho Nacional de Desenvolvimento Científico e Tecnológico (CNPq).
  Esse relatório se faz necessário por determinação do próprio CNPq, para concluir a vigência do contrato de bolsa.
  A principal atividade realizada neste período foi a obtenção do título de mestre pelo autor.

  O objetivo do mestrado foi implementar um sistema de localização em robôs humanoides autônomos.
  Para que robôs humanoides possam jogar futebol competitivamente de maneira autônoma é necessário que os robôs conheçam suas posições no campo, essa informação é essencial para o desenvolvimento de estratégias.
  A posição pode ser estimada a partir do conhecimento de como o robô se move pelo domínio e por observações feitas pelo próprio robô.
  Mas isso não é uma tarefa trivial.
  Os movimentos executados pelos robôs são imprecisos, além de problemas não modeláveis que surgem por problemas da parte física do robô.
  As observações feitas pelo robô são ruidosas, o que impedem que informações precisas de direção e distância sejam obtidas, pois seus poucos sensores estão em constante movimento devido ao balanço necessário para manter o robô em movimento.

  É comum encontrar trabalhos acadêmicos sobre localização de robôs autônomos para diversos domínios, mas são poucos os trabalhos que lidam com um domínio tão restrito quanto o deste trabalho.
  Além disso, os trabalhos sobre este domínio apresentam algoritmos e resultados que não são reprodutíveis, devido às diferenças de \textit{hardware} e \textit{software} dos robôs utilizados.
  Assim, este trabalho implementa um sistema de localização, baseado no algoritmo de localização de Monte-Carlo, para que robôs humanoides autônomos sejam capazes de estimar suas posições no domínio.
  
  O sistema implementado apresenta um método para estimar o quanto o robô se move ao longo do tempo e métodos diferentes para calcular quanto cada partícula representa a posição do robô real, além de métodos para se recuperar de erros de estimativa, para alterar a quantidade de partículas conforme o necessário e para estimar qual a melhor observação que o robô poderá fazer em instantes futuros.
  Foram realizados experimentos simulados e em robôs reais que validam os métodos implementados e mostram que os métodos propostos são eficientes para resolver o problema de localização.
  Por fim, trabalhos futuros incluem verificar o funcionamento do sistema em situação de jogo, além da expansão do sistema para um domínio genérico para observar o funcionamento dos métodos propostos e compará-las à outros métodos do estado da arte.

  \noindent
  \textbf{Palavras-chaves}: filtro de partículas. localização de monte-carlo. localização de robô. robô humanoide. futebol de robôs. relatório de atividades.
\end{resumo}
% ---

% ---
% inserir lista de ilustrações
% ---
\pdfbookmark[0]{\listfigurename}{lof}
\listoffigures*
\cleardoublepage
% ---

% ---
% inserir lista de tabelas
% ---
\pdfbookmark[0]{\listtablename}{lot}
\listoftables*
\cleardoublepage
% ---

% ---
% inserir lista de abreviaturas e siglas
% ---
\begin{siglas}
  \item[ABNT] Associação Brasileira de Normas Técnicas
  \item[abnTeX] ABsurdas Normas para TeX
  \item[CNPq] Conselho Nacional de Desenvolvimento Científico e Tecnológico
  \item[MCL] Localização de Monte-Carlo
  \item[FEI] Centro Universitário da Fundação Educacional Inaciana Pe. Sabóia de Medeiros
  \item[IAAA] Inteligência Artificial Aplicada à Automação
\end{siglas}
% ---

% ---
% inserir lista de símbolos
% ---
\begin{simbolos}
  \item[$ \Gamma $] Letra grega Gama
  \item[$ \Lambda $] Lambda
  \item[$ \zeta $] Letra grega minúscula zeta
  \item[$ \in $] Pertence
\end{simbolos}
% ---

% ---
% inserir o sumario
% ---
\pdfbookmark[0]{\contentsname}{toc}
\tableofcontents*
\cleardoublepage
% ---


% ----------------------------------------------------------
% ELEMENTOS TEXTUAIS
% ----------------------------------------------------------
\textual

\chapter{Introdução}

\chapter{Atividades Desenvolvidas}

\chapter{Resultados Obtidos}

\chapter{Considerações Finais}

% ----------------------------------------------------------
% Introdução (exemplo de capítulo sem numeração, mas presente no Sumário)
% ----------------------------------------------------------

\postextual

% ----------------------------------------------------------
% Referências bibliográficas
% ----------------------------------------------------------
\bibliography{bib}
\end{document}